\begin{abstract}
    This research proposes a Generative Segmentation Model for mobile robotics, prioritizing novel object recognition and robust handling of Out-of-Distribution (OoD) data. The objective is to enhance mobile robotic systems' navigation in unknown terrains by creating a model capable of resilient performance against OoD or, when necessary, signaling and triggering a fail-safe mode. Motivated by AI challenges in dynamic scenarios, the research integrates Uncertainty Quantification (UQ) to address Neural Networks' limitations with OoD data. UQ plays a crucial role in distinguishing certain and uncertain predictions, allowing for preemptive reactions or human intervention. Furthermore, UQ can be used for active learning, which reduces training costs.

    The research will compare Generative Segmentation Models with state-of-the-art (SOTA) models, by assessing robustness against OoD and novel object recognition. Furthermore, the effectiveness of UQ is assessed. Finally, if time permits, this research will investigate the possibility of efficiently incorporate temporal data into Denoising Diffusion Segmentation. In summary, this research outlines a concise plan for a generative Object Detection model, integrating Uncertainty Quantification to address challenges in mobile robotics and contributing to the advancement of AI applications in real-world scenarios.
    \noindent\textbf{Wordcount: 183}
\end{abstract}
