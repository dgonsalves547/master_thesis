\chapter{Introduction}\label{chapter:introduction}

Mobile robotics is a rapidly growing field that has numerous applications in industries such as logistics, agriculture, and healthcare \cite{cognominal2021evolving,kebede2024review,clark2023amazon}. With the increasing demand for efficient and autonomous mobile robots, researchers have been exploring various techniques to improve their performance. However, one of the major challenges faced by these robots is their limited ability to generalize to new environments and scenarios \cite{alatise2020review}.This can lead to poor performance and limited adaptability, highlighting the need for more effective unsupervised pretraining methods.

In recent years, machine learning is increasingly used in mobile robotics \cite{almeida2018localization,yu2018ds}. The current state of the art models are often pretrained on large-scale datasets and subsequently fine-tunded for the specific task and environment at hand \cite{Goodfellow-et-al-2016}. However, the data distribution encountered by robots in real-world scenarios is often vastly different from the one they were trained on, which can lead to poor performance and limited adaptability.

\subsection*{Research Questions}
This research aims to investigate whether unsupervised pretraining methods can be used to improve the performance of mobile robotics. It will explore how Variational AutoEncoders (VAEs) \cite{kingma2014autoencodingvariationalbayes} can be used to pretrain good feature extractors for semantic segmentation and the benefits the bayesian structure of the network has in terms of safety and robustness.

Specifically, it will answer the following questions
\begin{itemize}
    \item Does the encoder of a VAE produce useful features for the semantic segmenation task?
    \item Can VAEs be adapted to generate (semantic) segmentation masks from images, leveraging the unsupervised pretraining process?
    \item Does bootstrapping from the latent space of a variational encoder provide a reliable estimate of the confidence of the network for semantic segmentation tasks?
\end{itemize}


% An example use case that would be intresting for Avular -> One object is not recognized good enough or is of high importance that it should be recognized. Making images from the robot from various angles is tedious, however taking pictures with your phone (or handheld camera) is easier to do from various angles + you know that all pictures contain useful objects.
