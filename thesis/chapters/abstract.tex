\chapter*{Abstract}\label{chapter:abstract}

Mobile Robotics rely on Simultaneous Localization and Mapping (SLAM) to navigate in unknown and dynamic environments. SLAM works by detecting landmarks in the environment and relating the position of the robot to these landmarks. Semantic labels can be used to improve the detection of more robust landmarks. These semantic labels can be generated by Semantic Segmentation networks, however these require labelled data which are often expensive or hard to collect. The use of pre-trained models can reduce the amount of task specific data that is required, however these still require labelled data to train. This thesis investigates if Variational Auto-encoders can be used to do unsupervised pre-training for Semantic Segmentation.

The methodology involves the development of a VAE-based semantic segmentation architecture (VAES). This model is evaluated against the established baseline architectures, U-Net and Feature Pyramid Network (FPN), using the CoCo dataset. Additionally, it is examined how different architectural choices affect the performance and data requirements for semantic segmentation, with a focus on the impact of pre-trained weights, the backbone, and the overall architecture.

Our experimental results demonstrate that while the VAES model integrates variational inference with deep learning for segmentation, it does not yet surpass the simpler U-Net and FPN architectures in terms of the Jaccard Index. This outcome highlights the challenges associated with the additional complexity in the variational structure, particularly in resource-constrained environments typical of mobile robotics. However, our findings also emphasize the theoretical potential of pre-training in reducing the dependence on labelled data, pointing to avenues for future optimization and development.

These findings suggest that while the integration of variational inference techniques with deep learning-based segmentation offers theoretical benefits, practical applications in SLAM systems require further optimization to achieve competitive performance. 
