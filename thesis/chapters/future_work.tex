\chapter{Future Work}\label{chapter:future_work}

\section{Improving VAES}
As already stated in the background section~\ref{chapter:background}, the VAE has some flaws that have been shown to be mitigable. We will highlight some interesting research that could be adapted to our method to possibly improve the quality of predictions.

\subsection{Larger dataset}
In this thesis, the pre-training of the VAES has been limited to the COCO~\cite{lin2015microsoftcococommonobjects} ($\sim$100k) dataset. However, similar to transformers, it might be beneficial to pre-train on larger datasets, such as OpenImagesV7~\cite{OpenImages} ($\sim$9M).

\subsection{Improving the Backbone}
One of the drawbacks of Fully Convolutional Network (FCN), is that in their classic form, as used by us, they have a small Field of View. This means they cannot relate parts of the image that are far apart. This can partially be combatted by making use of dilated convolutional layers, as proposed by Gao~\cite{gao2023rethinking}, or by the use of vision transformers~\cite{dosovitskiy2021image}, which can relate pixels that are far apart in the input image by leveraging attention mechanisms. These vision transformers have been used successfully in the segmentation task~\cite{xie2021segformer,chen2022vision}. However, they have two major drawbacks. One, as transformers have a lower inductive bias for vision~\cite{dosovitskiy2021image} compared to convolutional networks, they require much more data and time to train. Secondly, their attention mechanisms are computationally more complex than convolutional layers, thus they are more difficult to optimize for edge devices such as those required for mobile robotics. The first drawback can be countered by using publicly available pre-trained models. The second problem is more fundamental, and although there is research into making transformers more efficient~\cite{li2022efficientformer,Yu2021MetaFormerIA}, they remain more computationally expensive compared to their CNN counterparts.

\subsection{Improving the Sharpness}
Some reconstructed samples clearly show the disappearing of key features for semantic segmentation. This problem could potentially be mitigated by using different loss functions that focus more on the details. Or by the use of better architectures, such as the H-VAEs discussed in Chapter~\ref{chapter:background}, which improve the sharpness and reconstruction capabilities of the VAE.

\section{Extending VAES}
Next to the possibilities to mitigate current problems, VAEs can also be extended to provide additional benefits compared to the more classic discriminative approaches. In this section, we will highlight some possible extensions.

\subsection{Anomaly detection}
VAEs have been shown to be able to provide rudimentary anomaly detection. There are two prominent methods for this. The first method is by detecting unusual latent variables from the approximate posterior output $q_{\phi}(z|x)$~\cite{marimont2020anomalydetectionlatentspace,angiulli2020improving,angiulli2023latent}. The second method is by calculating the reconstruction error from the given input~\cite{an2015variational, zhou2020unsupervisedanomalylocalizationusing, gouda2022unsupervised}. The main benefit of the first method is that the decoder is not required for anomaly detection, resulting in a lower inference cost. In the case of robotics, it is important to detect unusual situations as it might result in unexpected behaviour which in turn could lead to dangerous situations. Therefore, it would be interesting to see if our method is capable of providing these anomaly predictions as well.

Furthermore, a closely related use case is the ability to have an uncertainty estimation, by the use of bootstrapping~\cite{chen2018use,kohl2018probabilistic}, for the output of the model. The additional uncertainty information could then be used by downstream systems to be even more robust. Moreover, this information might be useful for various active learning~\cite{hino2020active} techniques. This allows for more effective labelling strategies, thus reducing the amount of labelling required and ensuring an efficient feedback loop from production environments. This could possibly be further combined with models like the Segment Anything Model (SAM)~\cite{kirillov2023segment}. It is a huge, and slow, network capable of, as the name implies, segmenting anything based on a prompt. This could aid in the labelling task, thus reducing the amount of human labour required to improve the models that are employed in the field.

\subsection{Multimodal encoder}
VAEs have been shown to be able to be trained for multimodal input~\cite{sadok2024multimodal,shi2019variational,sutter2021generalized,suzuki2016joint,wu2018multimodal}. Next to the image, the model also takes into account other forms of information (e.g., sound~\cite{sadok2024multimodal} or text~\cite{suzuki2016joint}). As extra sensors can easily be added to mobile robots, they might provide useful information to the latent space that could otherwise not be detected using visual queues.

Another additional input could be the past frames, as there are VAE-based video generation models~\cite{yan2021videogpt}. Especially when they are trained on datasets that require the model to understand object permanence. This would be extra beneficial in combination with feedback from the kinematics of the robot.
